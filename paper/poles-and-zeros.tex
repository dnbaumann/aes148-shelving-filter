%\section*{Appendix II: Zeros and Poles}
%
The zeros and poles of a 2nd order low-shelving filter are
\begin{align}
s_{0} = \omega_\mathrm{c} \:g_\mathrm{B}^{\pm\frac{1}{4}} \: \e^{\im\alpha}\qquad
s_{\infty} = \omega_\mathrm{c} \: g_\mathrm{B}^{\mp\frac{1}{4}} \: \e^{\im\alpha},
\end{align}
and the respective complex conjugates $s_{0}^{\ast}$ and $s_{\infty}^{\ast}$.
The sign in the exponents is again determined
by the shelving level $G_\mathrm{B}$.
The polar angle $\alpha$ in the complex $s$-plane (Laplace domain)
relates to
\begin{align}
Q = \frac{-1}{2\cos\alpha}.
\end{align}
For stable and causal filters,
the poles must be in the left-half $s$-plane $\Re(s_{\infty}) < 0$,
therefore $\frac{\pi}{2} < \alpha < \frac{3\pi}{2}$.
The corresponding $Q$-factor is always positive
and varies within $[\frac{1}{2}, \infty)$.
This paper mainly considers Butterworth \cite{Ballou2008} $Q=\frac{1}{\sqrt{2}}$
which leads to $\alpha=\frac{3}{4} \pi$
as illustrated in Fig.~\ref{fig:pzmap}.
If $Q=\frac{1}{2}$, i.e. $\alpha=\pi$,
the poles and zeros lie on the real axis,
thus double real poles $s_{0} = s_{0}^{\ast}$ and
$s_{\infty} = s_{\infty}^{\ast}$ are obtained.
%
In this case the 2nd order frequency
response is identical to that of the 1st order shelving filter, although
obtained by different pole/zero-configurations.
%
Thus, it is sufficient to discuss the present filter design based on
2nd order filters
%
\NewL In the case of high-shelving filters, the poles and zeros are exchanged. %,
%\begin{align}
%s_{0} &= \omega_{c,\mu} \: g_\mathrm{B}^{\mp\frac{1}{4}} \: e^{i\alpha}\\
%s_{\infty} &= \omega_{c,\mu} \: g_\mathrm{B}^{\pm\frac{1}{4}} \: e^{i\alpha}.
%\end{align}
Notice that a high-shelving filter with a shelving gain $\pm g_\mathrm{B}$
has the same poles and zeros as the low-shelving filter with $\mp g_\mathrm{B}$.
The system functions only differ by a scaling factor of $\pm g_\mathrm{B}$
due to the leading coefficients $b_{2}$ and $a_{2}$.
